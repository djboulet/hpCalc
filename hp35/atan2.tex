	\section{atan2}
	\setcounter{hppgline}{0}
	\subsection*{Description}
	This program calculates ${\rm atan2}(\frac{y}{x})$. Result is in the range $-180$\degree\ to $+180$\degree.
	\subsubsection*{Usage}
	\texttt{GTO Z001 \tab $x$ R/S \tab $y$ R/S }
	\subsection*{Program Listing}
	\hpline{Z}{LBL Z}{Start of program.}
	\hpline{Z}{STO X}{Store $x$ in \texttt{X}.}
	\hpline{Z}{STOP}{Wait for user R/S.}
	\hpline{Z}{STO Y}{Store $y$ in \texttt{Y}.}
	\hpline{Z}{RCL Y}{Recall \texttt{Y}. Note: this is also the entry point for subroutine.}
	\hpline{Z}{RCL X}{Recall \texttt{X}.}
	\hpline{Z}{$\div$}{Take ratio of rise over run ($\frac{y}{x}$).}
	\hpline{Z}{ATAN}{Calculate $\arctan(\frac{y}{x})$.}
	\hpline{Z}{STO R}{Save as an interim result in \texttt{R}.}
	\hpline{Z}{RCL X}{Test sign of \texttt{X}.}
	\hpline{Z}{$x > 0$?}{Is $x$ positive?}
	\hpline{Z}{GTO Z027}{If so then go to end of program.}
	\hpline{Z}{RCL Y}{Recall \texttt{Y} ...}
	\hpline{Z}{SGN}{Calculate its sign ...}
	\hpline{Z}{45}{}
	\hpline{Z}{$\times$}{then multiply it by 45\degree.}
	\hpline{Z}{RCL X}{Get \texttt{X} value.}
	\hpline{Z}{$x = 0$?}{Is it equal to zero?}
	\hpline{Z}{RTN}{If so then return the value of the stack ($\pm$45\degree)}
	\hpline{Z}{180}{Setup offset depending on sign of $y$.}
	\hpline{Z}{STO -R}{Initially subtract 180\degree --- we do this at a minimum.}
	\hpline{Z}{RCL Y}{Get \texttt{Y} value.}
	\hpline{Z}{$x < 0$?}{Is it negative?}
	\hpline{Z}{GTO Z027}{If yes, then we are done since we already subtracted 180\degree.}
	\hpline{Z}{360}{If $y$ is positive then we have to add 360\degree ...}
	\hpline{Z}{STO +R}{... for a total addition of 180\degree.}
	\hpline{Z}{RCL R}{Get the angle.}
	\hpline{Z}{RTN}{Return to calling function.}

	\subsection*{Example}
	In the following example we calculate ${\rm atan2}(\frac{+1.5}{-1.0})$ using the following steps: \newline	
	\hpexam{GTO Z001}{0.00000}{Go to start of program.}
	\hpexam{1.5}{1.5}{Your value for $x$.}
	\hpexam{R/S}{1.50000}{}
	\hpexam{-1.0}{-1.0}{Your value for $y$.}
	\hpexam{R/S}{-33.69007}{The resulting angle.}

	\subsection*{Comments}
	Users have to be careful about a couple of things:
	\begin{enumerate}
	\item Angles are calculated in \it{degrees}\rm.  Confirm calculator setting before using this function.
	\item User is responsible for ensuring that $x$ and $y$ are \bf{never}\rm\ both zero.
	\end{enumerate}
	
