	\section{Modular Exponentiation}
	\setcounter{hppgline}{0}
	\subsection*{Description}
	This program calculates the modulus of a number raised to a large power.  The formula looks like this:
	\[ {modexp} = n^p\ \bmod m\]
	\subsubsection*{Usage}
	\texttt{GTO A001 \tab $n$ R/S \tab $p$ R/S \tab $m$ R/S}
	\subsection*{Program Listing}
	\hpline{A}{LBL A}{start of program}
	\hpline{A}{STO N}{store number to the raised to the power P}
	\hpline{A}{STOP}{wait for user R/S}
	\hpline{A}{STO P}{store exponent}
	\hpline{A}{STOP}{wait for user R/S}
	\hpline{A}{STO M}{store modulus}
	\hpline{A}{1}{initialize product ...}
	\hpline{A}{STO R}{... and save in memory}
	\hpline{A}{RCL N}{recall base ...}
	\hpline{A}{RCL R}{recall product ...}
	\hpline{A}{$\times$}{... and multiply the two}
	\hpline{A}{RCL M}{recall the modulus ...}
	\hpline{A}{RMDR}{... and apply it}
	\hpline{A}{STO R}{save the new product}
	\hpline{A}{DSE P}{decrement exponent ...}
	\hpline{A}{GTO A009}{... and loop back if not finished.}
	\hpline{A}{RCL R}{pull the product from memory}
	\hpline{A}{RTN}{we are done!}
	\subsection*{Example}
	In the following example we calculate $\mathbf{5^{101} \bmod 31}$ using the following steps: \newline	
	\hpexam{GTO A001}{0.00000}{go to start of program}
	\hpexam{5 R/S}{5.00000}{the ``base''}
	\hpexam{101 R/S}{101.00000}{the ``exponent''}
	\hpexam{31 R/S}{25.00000}{the ``modulus'' and result}

	\subsection*{Comments}
	The HP35s is not known for it's lightning speed.  The above example
	will take about 12 seconds to run.
